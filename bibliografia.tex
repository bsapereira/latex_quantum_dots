% INTRODUÇÃO
\bibitem{introducao1} de Figueiredo Oliveira, André Rezende. \href{http://www.infis.ufu.br/sites/infis.ufu.br/files/Anexos/Bookpage/TCC%20F%C3%8DSICA%20DE%20MATERIAIS%202009_2%20-%20ANDRE%20REZENDE.pdf}
{\it Caracterização Óptica de Pontos Quânticos Semicondutores de CdS em Matrizes Poliméricas}. 
\bibitem{introducao2} Ekimov, Alexey I., and Alexei A. Onushchenko. \href{https://www.researchgate.net/profile/Alexey_Onushchehko/publication/234289541_Quantum_Size_Effect_in_Three-Dimensional_Microscopic_Semiconductor_Crystals/links/0c9605305fa93c4e3d000000.pdf}{\it Quantum size effect in three-dimensional microscopic semiconductor crystals}. Jetp Lett 34.6 (1981): 345-349.
\bibitem{introducao3} Brus, Louis E. \href{http://aip.scitation.org/doi/10.1063/1.447218}{\it Electron-electron and electron-hole interactions in small semiconductor crystallites: The size dependence of the lowest excited electronic state}. The Journal of chemical physics 80.9 (1984): 4403-4409.
\bibitem{introducao4} Murray, CBea, David J. Norris, and Moungi G. Bawendi. \href{http://pubs.acs.org/doi/abs/10.1021/ja00072a025?journalCode=jacsat}{\it Synthesis and characterization of nearly monodisperse CdE (E= sulfur, selenium, tellurium) semiconductor nanocrystallites}. Journal of the American Chemical Society 115.19 (1993): 8706-8715.
\bibitem{introducao5} Zhu, Jun-Jie, et al. \href{https://link.springer.com/book/10.1007/978-3-642-44910-9}{\it Quantum dots for DNA biosensing. Vol. 165}. Springer, 2013.

% INTRODUÇÃO A MECÂNICA QUÂNTICA E FÍSICA DO ESTADO SÓLIDO
  %Redes Cristalinas
\bibitem{qm_fis0} Koole, Rolf, et al. \href{https://link.springer.com/chapter/10.1007/978-3-662-44823-6_2}{\it Size effects on semiconductor nanoparticles}. Nanoparticles. Springer Berlin Heidelberg, 2014. 13-51.
\bibitem{qm_fis1} Band, Yehuda B., and Yshai Avishai. \href{https://www.elsevier.com/books/quantum-mechanics-with-applications-to-nanotechnology-and-information-science/band/978-0-444-53786-7}{\it Quantum mechanics with applications to nanotechnology and information science. Academic Press, 2013}.
\bibitem{qm_fis2} Oliveira, Ivan S., and Vitor LB De Jesus. \href{http://www.saraiva.com.br/introducao-a-fisica-do-estado-solido-2-ed-3527998.html}{\it Introdução à física do estado sólido}. Editora Livraria da Fisica, 2005.
\bibitem{heterojuncao} Margaritondo, Giorgio, ed. \href{http://www.springer.com/br/book/9789027728234}{\it Electronic structure of semiconductor heterojunctions}. Vol. 1. Springer Science \& Business Media, 2012.
  %Equação de Schrödinger
\bibitem{qm_fis3} Eisberg, Robert Martin; Resnick, Robert - \href{https://www.livrariadafisica.com.br/detalhe_produto.aspx?id=6237}{\it Física Quântica}
\bibitem{qm_fis4} C.W. Sherwin- \href{https://www.amazon.com/Introduction-Quantum-Mechanics-C-W-Sherwin/dp/0030068851}{\it Introduction to quantum mechanics} - Holt, Rinehart \& Winston of Canada Ltd (1959). 
\bibitem{qm_fis5} Ashcroft, Neil W., and N. David Mermin. \href{https://www.livrariadafisica.com.br/detalhe_produto.aspx?id=101883}{\it Física do estado sólido}. Cengage Learning, 2011.
\bibitem{qm_fis6} Kittel, Charles. \href{https://www.amazon.com.br/dp/8521615051/ref=asc_df_85216150515016109?smid=A1ZZFT5FULY4LN&tag=goog0ef-20&linkCode=asn&creative=380341&creativeASIN=8521615051}{\it Introdução à física do estado sólido}. Grupo Gen-LTC, 2006.
\bibitem{qm_fis7} Thompson, D. \href{http://aapt.scitation.org/doi/abs/10.1119/1.18243}{\it The reciprocal lattice as the Fourier transform of the direct lattice}. American Journal of Physics 64.3 (1996): 333-334.
\bibitem{qm_fis8} Wang, Gwo-Ching, and Toh-Ming Lu. \href{https://books.google.com.br/books?hl=pt-BR&lr=&id=LEC9BAAAQBAJ&oi=fnd&pg=PR6&dq=ISBN:+978-1-4614-9286-3&ots=6N688QVTZa&sig=eKxt-la7NjrX5ZWrmHEoIZpaTwQ#v=onepage&q=ISBN%3A%20978-1-4614-9286-3&f=false}{\it RHEED Transmission Mode and Pole Figures: Thin Film and Nanostructure Texture Analysis}. Cap. 2. Springer Science \& Business Media, 2013.
\bibitem{qm_fis9} I. Prigogine, Stuart A. Rice. \href{http://onlinelibrary.wiley.com/book/10.1002/9780470142813}{\it Advances in Chemical Physics, Volume 57}. Wiley Interscience, 2007.
\bibitem{qm_fis10} Dagotto, Elbio. \href{http://www.springer.com/la/book/9783540432456} {\it Nanoscale Phase Separation and Colossal Magnetoresistance: The Physics of Manganites and Related Compounds}. Springer-Verlag Berlin Heidelberg, 2003 
\bibitem{qm_fis11} Rodrigues, Rafael de Lima. \href{http://sbfisica.org.br/rbef/pdf/v19_68.pdf}{\it Mecânica Quântica na Descrição de Schrödinger}. Revista Brasileira de Ensino de Física, vol. 19, n 1, p.68-83, 1997

% FRUSTRADO - TODO: Mudar links
 \bibitem{frustrado1} Eisberg, Robert Martin Resnick, and Cota Araiza Robert. \href{http://gen.lib.rus.ec/book/index.php?md5=80CCC290E6FFF645ADF0BA24178E4C5D}{\it Física cuántica: átomos, moléculas, sólidos, núcleos y partículas. 1994}.
 \bibitem{frustrado2} Atkins, Peter, and Loretta Jones. \href{http://gen.lib.rus.ec/book/index.php?md5=6D32E94CECA0A9BD6FFF5F1307078071}{\it Chemical principles: The quest for insight. Macmillan, 2007}.
 \bibitem{frustrado3} Peter, Y. U., and Manuel Cardona. \href{http://gen.lib.rus.ec/book/index.php?md5=20A8507AB491C812ED2C75D08740987A}{\it Fundamentals of semiconductors: physics and materials properties}. Springer Science \& Business Media, 2010.
 \bibitem{frustrado5} Dagotto, Elbio. \href{http://gen.lib.rus.ec/book/index.php?md5=3C621FEBFE1EBBF8B376CED188D04A84}{\it Nanoscale phase separation and colossal magnetoresistance: the physics of manganites and related compounds. Vol. 136}. Springer Science \& Business Media, 2013.
 \bibitem{frustrado6} de Lima Rodrigues, Rafael. \href{http://sbfisica.org.br/rbef/pdf/v19_68.pdf}{\it Mecânica Quântica na Descrição de Schrödinger.} Revista Brasileira de Ensino de Física 19.1, 1997.
 \bibitem{frustrado7} Zill, Dennis G., and Michael R. Cullen. \href{http://gen.lib.rus.ec/book/index.php?md5=8673E58CC84FED5F909BEA1CC2BC4E3F}{\it Differential Equations With Boundary-Value Problems}. Thomson Brooks/Cole, 1996.
 % Potencial Periódico e Teorema de Bloch
 \bibitem{bloch1} Silva, Jusciane da Costa e. \href{www.repositorio.ufc.br/bitstream/riufc/12669/1/2008_tese_jcsilva.pdf}{\it Confinamento Quântico em Hetero-estruturas Semicondutoras de Baixa Dimensionalidade}. Universidade Fedaral do Ceará, 2008.
 \bibitem{bloch2} H. Ibach and Hans Lüth, \href{http://www.springer.com/us/book/9783540938033}{\it Solid-State Physics: An Introduction to Principles of Materials Science}, Springer-Verlag, 2nd Ed., 1995
% Semicondutores Bulk
\bibitem{bulk1} Averill, B. A., and P. Eldredge. \href{https://2012books.lardbucket.org/books/principles-of-general-chemistry-v1.0m/index.html}{\it Principles of General Chemistry (v. 1.0)}. Creative Commons licensed (2012): 2991.
\bibitem{bulk2} Sattler, Klaus D. \href{https://www.amazon.com/Handbook-Nanophysics-Nanoparticles-Quantum-Dots-ebook/dp/B008I9VLAI}{\it Handbook of Nanophysics: Nanoparticles and Quantum Dots}. CRC Press, 2016.
%Confinamento
\bibitem{confinamento1} Yoffe, A. D. \href{http://adsabs.harvard.edu/abs/1993AdPhy..42..173Y}{\it Low-dimensional systems: quantum size effects and electronic properties of semiconductor microcrystallites (zero-dimensional systems) and some quasi-two-dimensional systems}. Advances in Physics, vol. 42, Issue 2, p.173-262, 1993
\bibitem{confinamento2} Dick, Rainer. \href{http://www.springer.com/us/book/9781489990686}{\it Advanced Quantum Mechanics: Materials and Photons}. Springer-Verlag New York, 2012
\bibitem{confinamento3} Harrison, Paul \href{http://www.wiley.com/WileyCDA/WileyTitle/productCd-0470010819.html}{\it Quantum Wells, Wires and Dots: Theoretical and Computational Physics of Semiconductor Nanostructures}. Wiley, 2005
\bibitem{confinamento4} Ephrem O. Chukwuocha, Michael C. Onyeaju, Taylor S. T. Harry. \href{http://file.scirp.org/pdf/WJCMP20120200011_36451105.pdf}{\it Theoretical Studies on the Effect of Confinement on Quantum Dots Using the Brus Equation}. World Journal of Condensed Matter Physics, 2012, 2, 96-100.
\bibitem{confinamento5} Maronesi, Ray Nascimento. \href{http://www.locus.ufv.br/handle/123456789/9775}{\it Nova técnica para síntese de pontos quânticos coloidais de CdS em meio puramente aquoso}. Universidade Federal de Viçosa, 2016
%Aplicações Opticas
\bibitem{optica1} Melville, Jonathan. \href{https://www.ocf.berkeley.edu/~jmlvll/lab-reports/quantumDots/quantumDots.pdf}{\it Optical Properties of Quantum Dots}. UC Berkeley College Of Chemistry, 2015
\bibitem{optical2} Jones, Aaron, and Nick Verlinden. \href{https://web.wpi.edu/Pubs/E-project/Available/E-project-042607-125225/unrestricted/QuantumDots.pdf}{\it Optical Properties of Quantum Dots: An Undergraduate Physics Laboratory}. 2007.
%Aplicações Sintese
\bibitem{sintese1} Machado, Claudia Emanuele, et al. \href{https://www.researchgate.net/profile/Marco_Schiavon2/publication/279939933_Carbon_Quantum_Dots_Chemical_Synthesis_Properties_and_Applications/links/559fe9f908aed84bedf44826.pdf}{\it Pontos Quânticos de Carbono: Síntese Química, Propriedades e Aplicações}. Revista Virtual de Química 7.4 (2015): 1306-1346.
\bibitem{sintese2} Wang, Youfu, and Aiguo Hu. \href{http://www.rsc.org/chemical-sciences-repository/articles/article/dr000000002036?doi=10.1039%2Fc4tc00988f}{\it Carbon quantum dots: synthesis, properties and applications}. Journal of Materials Chemistry C 2.34 (2014): 6921-6939.
\bibitem{sintese3} Wakaoka, Takuo, et al. \href{http://pubs.rsc.org/en/content/articlelanding/2014/tc/c4tc01136h#!divAbstract}{\it Confined synthesis of CdSe quantum dots in the pores of metal–organic frameworks}. Journal of Materials Chemistry C 2.35 (2014): 7173-7175.
\bibitem{sintese4} da Silva, Isaías Ferreira. \href{http://www.dsif.fee.unicamp.br/~furio/IE607A/Pl.pdf}{\it Medidas de Caracterização e Análise de Materiais: Espectroscopia De Fotoluminescência}. 2000.
\bibitem{sintese5} Neto, Ernesto Soares de Freitas. \href{https://repositorio.ufu.br/bitstream/123456789/15607/1/ErnestoSoares.pdf}{\it Estudo de Pontos Quânticos Semicondutores e Semimagnéticos}. Universidade Federal de Uberlândia, 2013.
\bibitem{sintese6} Rodrigues, Ariano De Giovanni., José Cláudio Galzerani. \href{http://www.scielo.br/scielo.php?script=sci_arttext&pid=S1806-11172012000400009&lng=pt&tlng=pt}{\it Espectroscopias de infravermelho, Raman e de fotoluminescência: potencialidades e complementaridades}. Rev. Bras. Ensino Fís. vol.34 no.4 São Paulo out. 2012.
\bibitem{sintese7} A. Balandin, K. L. Wang, N. Kouklin, S. Bandyopadhyay. \href{http://sci-hub.cc/10.1063/1.125681}{\it Raman spectroscopy of electrochemically self-assembled CdS quantum dots}. Applied Physics Letter Volume76, Number 2, 2000.
\bibitem{sintese8} Fontes, Adriana, and Beate Saegesser Santos, eds. \href{http://www.springer.com/gp/book/9781493912797}{\it Quantum dots: applications in biology}. Springer New York, 2014.
\bibitem{sintese9} Raele, Renata Almeida. \href{http://repositorio.ufpe.br/bitstream/handle/123456789/12231/Disserta\%C3\%A7ao\%20Renata\%20Raele.pdf?sequence=1\&isAllowed=y}{\it AVALIAÇÃO DA CITOTOXICIDADE DE QUANTUM DOTS, IN VITRO, EM CÉLULAS RAW 264.7}. Universidade Federal de Pernambuco, 2013.
\bibitem{sintese10} Chen, Jian, Veeral Hardev, and Jeff Yurek. \href{https://www.researchgate.net/publication/291313785_Quantum-dot_displays_Giving_LCDs_a_competitive_edge_through_color}{\it Quantum Dot Displays: Giving LCDs a Competitive Edge Through Color}. Nanotech. L. \& Bus. 11 (2014): 4.
\bibitem{sintese11} Talapin, Dmitri V., and Jonathan Steckel. \href{https://www.cambridge.org/core/journals/mrs-bulletin/article/quantum-dot-lightemitting-devices/94FDE7F4EF98D19F4B78E2AD2F656C4F}{\it Quantum dot light-emitting devices}. Mrs Bulletin 38.09 (2013): 685-691.
\bibitem{sintese12} Han, Hau-Vei, et al. \href{https://www.osapublishing.org/oe/abstract.cfm?uri=oe-23-25-32504}{\it Resonant-enhanced full-color emission of quantum-dot-based micro LED display technology}. Optics express 23.25 (2015): 32504-32515.