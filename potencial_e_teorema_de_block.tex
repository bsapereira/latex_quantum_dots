\par Para entender as propriedades eletrônicas dos pontos quânticos, uma análise de semicondutores \textit{bulk} (não confinados) é necessária. Como calcular o estado eletrônico para esses materiais é algo complexo, utilizam-se aproximações. Umas dessas aproximações consiste na observação do comportamento de um elétron, assumindo que todos os outros fazem parte dos íons que criam o potencial periódico.

\subsubsection{Potencial Periódico}

	\par A hamiltoniana\cite{qm_fis5} de um sólido contém tanto os potenciais monoeletrônicos quanto os potenciais de par. O primeiro descreve as interações dos elétrons com os núcleos e o segundo, as interações entre os elétrons. 

	\par Para o elétron independente, essas interações são representadas por um potencial efetivo monoeletrônico $U(\mathbf{r})$. Independente da forma desse potencial, se o cristal for perfeitamente periódico, ele deve satisfazer a seguinte equação:

	\begin{equation}
		\label{bloch_1}
		U(\mathbf{r}+\mathbf{R}) = U(\mathbf{R}).
	\end{equation}

	Para o caso do elétron livre, $U(\mathbf{R})$ é zero, caracterizando o caso mais simples de um periódico. Assim, a equação de Schrödinger para o elétron livre se reduz à equação \eqref{bloch_1}. O vetor \textbf{r} está relacionado à célula primitiva, enquanto \textbf{R} indica os pontos de uma rede de Bravais\cite{qm_fis2}.

\subsubsection{Teorema de Bloch}

	\par O teorema de Bloch\cite{qm_fis5} afirma que os autoestados $\Psi$ podem ser escritos como uma onda plana vezes uma função de periodicidade para elétrons independentes, considerando que sobre eles atua um potencial periódico em uma rede de Bravais

	\begin{equation}
		\label{bloch_2}
		\Psi_{nk}(\mathbf{r})= e^{i\mathbf{k} \cdot \mathbf{r}}\cdot u_{nk}(\mathbf{r}).
	\end{equation}

	As equações \eqref{bloch_1} e \eqref{bloch_2} implicam em:

	\begin{align}
		\label{bloch_3}
		\Psi_{nk}(\mathbf{R} + \mathbf{r}) &= e^{i\mathbf{k} \cdot \mathbf{r}}\cdot u_{nk}(\mathbf{R} + \mathbf{r})\\
		\Psi(\mathbf{R} + \mathbf{r}) &= e^{i\mathbf{k} \cdot \mathbf{R}}\cdot \Psi(\mathbf{r}).
	\end{align}

	\par Os subíndices $n$ e $k$ serão explicados no final da prova do Teorema de Bloch, mas sua omissão não comprometará o entendimento do desenvolvimento a seguir.


	\par \textbf{- Condição de Contorno de Born-Von Karman}

		\par Serão aplicadas as condições de contorno para que as funções de onda sejam periódicas.
		
		\par No caso tridimensional,

		\begin{align}\label{bloch_4}
	        \left\{
	          \begin{array}{ll}
	            \displaystyle \Psi(x+L, y, z) &= \Psi(x, y, z)\\
	            \displaystyle \Psi(x, y+L, z) &= \Psi(x, y, z)\\
	            \displaystyle \Psi(x, y, z+L) &= \Psi(x, y, z)
	          \end{array}
	        \right.
	        .
	      \end{align}
		
		\par A equação \eqref{bloch_4} é conhecida como condição de contorno de Born-Von Karman de periodicidade macroscópica.  Como nem sempre a rede de Bravais é cúbica, generaliza-se a condição de contorno para:

		\begin{equation}
			\label{blochh_5}
			\Psi(\mathbf{r} + N i\cdot \mathbf{a}_{i}) = \Psi(\mathbf{r}),\ i=1,2,3\ ,
		\end{equation}
		no qual $\mathbf{a}_{i}$ são os vetores primitivos da rede direta e $Ni$ são números inteiros, de forma que:

		\begin{equation}
			\label{bloch_6}
			N1 \cdot N2 \cdot N3 = N,
		\end{equation}
		onde N é o número total de células primitivas no cristal.

		\par Aplicando o Teorema de Bloch, obtém-se:

		\begin{equation}
			\label{bloch_7}
			\Psi(\mathbf{r} + Ni \cdot \mathbf{a}_{i}) = e^{i\cdot N i \cdot \mathbf{k} \cdot \mathbf{a}_{i}} \cdot \Psi(\mathbf{r}).
		\end{equation}

		\par A equação \eqref{bloch_7} será válida para:

		\begin{equation}
			\label{bloch_8}
			e^{i\cdot N i \cdot \mathbf{k} \cdot \mathbf{a}_{i}} = 1,\ i=1,2,3.
		\end{equation}

		\par A partir do mesmo desenvolvimento feito em \eqref{redeReciproca_eq6},

		\begin{equation}
			\label{bloch_9}
			\mathbf{k} \cdot \mathbf{a}_{i} = 2 \pi \mathbf{k} i.
		\end{equation}

		\par Substituindo \eqref{bloch_9} em \eqref{bloch_8}, tem-se que:

		\begin{equation}
			\label{bloch_10}
			e^{i\cdot N i 2 \pi \cdot \mathbf{k} i} = 1.
		\end{equation}

		\par Uma exponencial complexa pode ser escrita em função de senos e cossenos, através da fórmula de Euler. Observa-se que a relação acima será válida sempre que $2\pi$ estiver sendo multiplicado por um número inteiro. Como $Ni$ é um número inteiro e $\mathbf{k}i$ também, pode-se definir $\mathbf{k}i$ em função de $Ni$:

		\begin{equation}
			\label{bloch_a}
			ki = \frac{mi}{Ni},\ com\ mi\ inteiro.
		\end{equation}

		\par Assim, a forma geral dos vetores de onda de Bloch permitidos é:


		\begin{equation}
			\label{bloch_11}
			\mathbf{k} = \sum_{i=1}^3 \frac{mi}{Ni} \cdot \mathbf{b}i.
		\end{equation}

		\par É definido que o elemento de volume, ou seja, o menor volume $\Delta\mathbf{k}$ da rede recíproca é formado pelo paralelepípedo com arestas $\frac{bi}{Ni}$, que pode ser escrito pelo produto misto:

		\begin{equation}
			\label{bloch_12}
			\Delta \mathbf{k} = \frac{\mathbf{b1}}{N1} \cdot \left( \frac{\mathbf{b2}}{N2} x \frac{\mathbf{b3}}{N3} \right)
				= \frac{1}{N} \cdot \mathbf{b1} \cdot \left(\mathbf{b2} x \mathbf{b3}\right).
		\end{equation}

		\par O vetor de onda pode ser escrito como $k=\frac{2\pi}{a}$, em que a é o parâmetro de rede da célula unitária. Assim, obtém-se $k^3=\frac{8\pi^3}{a^3}$, ou seja, $k^3=\frac{8\pi^3}{v}$, em que $v$ é o volume de uma célula primitiva na rede direta. Esse volume também é dado por $v=\frac{V}{N}$, onde $V$ representa o volume associado ao número de sítios da rede. Como o produto misto dos vetores $\mathbf{bi}$ representa o volume $k^3$. Logo, a equação \eqref{bloch_12} pode ser reescrita por:

		\begin{equation}
			\label{bloch_13}
			\Delta \mathbf{k} = \frac{8\pi^3}{V}.
		\end{equation}

		\par Ao construir o teorema de Bloch para esse volume, ele se torna válido para toda a rede cristalina devido a sua periodicidade.

	\par \textbf{- Prova do Teorema de Bloch}

	\par Para realizar a prova desse teorema, parte-se do princípio de que qualquer função que atenda às condições de Born-Von Karman pode ser expandida em ondas planas, sendo que essa expansão é feita por séries de Fourier. 
	
	\par Expandindo o potencial com periodicidade na rede de Bravais e a função de onda em ondas planas, obtém-se, respectivamente: 

	\begin{align}\label{bloch_14}
	      \begin{array}{ll}
	        \displaystyle U(\mathbf{r}) &= \sum_{\mathbf{K}} U_{\mathbf{K}} e^{i\mathbf{K}\cdot \mathbf{r}}\\
	        \displaystyle \Psi(\mathbf{r}) &= \sum_{\mathbf{K}} C_{\mathbf{k}} e^{i\mathbf{k}\cdot \mathbf{r}}	            
	      \end{array}
	      ,
  \end{align}

	onde $C_{\textbf{k}}$ e $U_{\textbf{K}}$ são coeficientes.

	\par Assim, tem-se

	\begin{align}\label{bloch_15}
	      \begin{array}{ll}
	        \displaystyle \frac{\partial \Psi(\mathbf{r})}{\partial\mathbf{r}} &= \sum_{\mathbf{K}} C_{\mathbf{k}} ik e^{i\mathbf{k}\cdot \mathbf{r}}\\
	        \displaystyle \frac{\partial^2 \Psi(\mathbf{r})}{\partial\mathbf{r}^2} &= -\sum_{\mathbf{K}} C_{\mathbf{k}} k^2 e^{i\mathbf{k}\cdot \mathbf{r}}	            
	      \end{array}
	      .
  \end{align}

	Substituindo as equações de \eqref{bloch_15} na equação de Schrödinger \eqref{eq_schrodinger_frustrado}, tem-se que

	\begin{equation}
		\label{bloch_16}
		\sum_{\mathbf{K}} C_{\mathbf{k}} k^2 e^{i\mathbf{k}\cdot \mathbf{r}} \cdot \frac{\hbar}{2m}
			+ \sum_{\mathbf{K}} C_{\mathbf{k}} ik e^{i\mathbf{k}\cdot \mathbf{r}}
			= E \cdot \sum_{\mathbf{K}} C_{\mathbf{k}} e^{i\mathbf{k}\cdot \mathbf{r}}.
	\end{equation}

	Reorganizando os termos para que se consiga colocar $e^{i\mathbf{k}\cdot \mathbf{r}}$ em evidência:

	\begin{equation}
		\label{bloch_17}
		\sum_{\mathbf{k}} e^{i\mathbf{k}\cdot \mathbf{r}}
			\left[ 
				C_\mathbf{k} \left( \frac{k^2 \hbar^2}{2m} - E \right)
				+ \sum_{\mathbf{K}} \mathbf{U}_{\mathbf{K}} \cdot C_{\mathbf{k-K}}
		   \right] = 0.
	\end{equation}

	Como a equação \eqref{bloch_17} é válida para todo $\mathbf{r}$, então

	\begin{equation}
		\label{bloch_18}
		\left( \frac{k^2 \hbar^2}{2m} - E \right)
				+ \sum_{\mathbf{K}} \mathbf{U}_{\mathbf{K}} \cdot C_{\mathbf{k-K}} = 0.
	\end{equation}

	\par A equação \eqref{bloch_18} é análoga à equação de Schrödinger só que para o espaço recíproco\cite{qm_fis2}. Um resultado importante obtido a partir dessa equação é que o coeficiente $C_{\mathbf{k}}$ se relaciona com os demais coeficientes\cite{qm_fis5}. Ou seja, $C\mathbf{k}$ acopla-se com $C_{\mathbf{k-K1}}$, $C_{\mathbf{k-K2}}$, etc. Isso significa que um ponto $C_{\mathbf{k}}$ da rede sofre influência de todos os outros pontos, gerando uma sobreposição das funções de onda. Isso fornece uma introdução aos níveis de energia\cite{bloch2}. É interessante observar que para $U_{\mathbf{K}} = 0$, a energia obtida é a do eĺétron livre.

	\par Pode reescrever a função de onda como:

	\begin{equation}
		\label{bloch_19}
		\Psi (\mathbf{r}) = \sum_{K} C_{\mathbf{k-K}}\cdot e^{i\mathbf{(k-K)}\cdot\mathbf{r}}.
	\end{equation}

	\par Trazendo $e^{i\mathbf{k} \cdot \mathbf{r}}$ para fora do somatório:

	\begin{equation}
		\label{bloch_20}
		\Psi (\mathbf{r}) = e^{i\mathbf{k}\cdot\mathbf{r}} \cdot \left(\sum_{K} C_{\mathbf{k-K}}\cdot e^{-i \mathbf{K}\cdot\mathbf{r}}\right).
	\end{equation}

	\par Da equação \eqref{bloch_20}, tem-se que a função de periodicidade $u$ é dada por:

	\begin{equation}
		\label{bloch_21}
		u(\mathbf{r}) = \sum_{K} C_{\mathbf{k-K}}\cdot e^{-i \mathbf{K}\cdot\mathbf{r}}.
	\end{equation}

	\par Assim, provou-se o teorema de Bloch, ao encontrar uma função de periodicidade $u$ que satisfaça a equação \eqref{bloch_1}.

	\par Como o problema foi analisado para um volume fixo, que diz respeito à região da célula unitária (equação \eqref{bloch_13}), tem-se infinitas soluções para k  com autovalores discretos, indexados pelo índice de bandas $n$. As energias $En(\mathbf{K})$ varia de forma contínua à medida que $\mathbf{k}$ varia e representam os níveis de energia para o elétron em um potencial periódico. Em relação a $n$, a energia varia de forma discreta. Ao atribuir os índices $n$ para as funções de onda e para os níveis de energia, obtém-se e $E n\mathbf{k}$. As informações que essas funções contêm são chamadas de estrutura de banda. Para cada $n$, o conjunto de níveis eletrônicos é chamado de banda de energia\cite{qm_fis5}. 

