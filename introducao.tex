\par Pontos quânticos\cite{introducao1} são nanopartículas semicondutoras, que por terem o tamanho reduzido, se comportam como se estivesse sujeitas a um poço de potencial. Isso faz com que elétrons e buracos sofram um forte confinamento quântico nas três dimensões espaciais. Devido a esse confinamento, eles têm sua energia quantizada em valores discretos, como em um átomo. Por isso também são chamados de átomos artificiais.

\par A história dessas nanopartículas começa em 1980 em um artigo publicado pelo físico russo, Ekimov. Seu trabalho foi baseado no estudo de efeitos quânticos de tamanho em cristais semicondutores microscópicos em três dimensões. Segundo Ekimov \cite{introducao2}:

\begin{quote}
\textit{"Apresentamos a descoberta e a espectroscopia de uma nova classe de objetos que exibem efeitos quânticos de tamanho: cristais microscópicos tridimensionais de compostos semicondutores crescidos em uma matriz dielétrica transparente."}
\end{quote}

\par Apesar de tratar como uma nova classe de objetos, Ekimov não os denomida nem os trata como pontos quânticos.

\par Durante seu estudo, ao realizar a absorção espectral de três amostras de cristais microscópicos com raios diferentes, observou uma dependência na posição espectral das linhas de absorção, onde supôs se tratar de um efeito quântico de tamanho ocorrido devido ao tamanho da partícula, e complementa:

\begin{quote}
\textit{“Tais portadores de carga na matriz dielétrica estão presos em um poço de potencial onde as paredes são limitadas pelo cristal microscópico.”}
\end{quote}

\par Apesar de considerar esse poço de potencial simétrico e esférico, Ekimov não se aprofunda no tópico, deixando a discussão dos pontos quânticos como "um efeito quântico".

\par Foi apenas em 1984 que Louis Brus, ao apresentar uma relação entre o tamanho e o \textit{gap} de energia das nanopartículas semicondutoras após aplicar um modelo esférico na função de onda de um semicondutor Bulk para uma partícula, começa a dar forma ao conceito que futuramente será denominado ponto quântico.

\par No artigo\cite{introducao3}, Brus considera cristalitos suficientemente pequenos de modo que os níveis de energia Bulk não sejam válidos e afirma:

\begin{quote}
\textit{“Conforme os cristalitos se aproximam do tamanho da excitação 1S, as interações elétron-buraco com a superfície destes cristalitos dominam a dinâmica. Neste limite molecular, a energia dependerá do tamanho e forma dos cristalitos e da natureza do material.”}
\end{quote}

\par Apesar dos avanços nos estudos de Brus, levou cerca de uma década para que o ponto quântico voltasse a ser estudado, quando Murray\cite{introducao4}, durante seu estudo sobre evolução de propriedades de materiais pelo tamanho de cristalitos nanométricos, afirma que o regime de tamanhos intermediários dos cristalitos afetam o comportamento de materiais Bulk, emergindo a natureza discreta das propriedades moleculares deles.

\par O objetivo de Murray era estudar estes diferentes regimes para observar e, se possível, controlar certos comportamentos como efeitos ópticos de estados excitados altamente polarizados ou comportamento fotoquímico.

\par Já era conhecido que as propriedades físicas desses nanocristais semicondutores eram regidas por confinamentos espaciais de excitação e, apesar do progresso de certos semicondutores, a interpretação dos dados experimentais era complicada de ser analisada devido a problemas na superfície derivacional e baixa cristalinidade destes materiais.
     
\par Murray se concentrou na síntese e caracterização de nanocristais semicondutores de CdSe, devido às propriedades ópticas e eletroquímicas deste material, resultando futuramente em sínteses coloidais de compostos CdX (X= S, Se, Te) que permearam os estudos em pontos quânticos e suas aplicações.

\par As aplicações destes compostos foram se tornando cada vez mais amplas com o passar do tempo, porém, devido a toxicidade dos íons de Cádmio, ainda era inviável sua utilização na área biomédica.

\par A fim de melhorar a estabilidade dos núbleos de nanocristais para redução da toxicidade, passou-se a introduzir camadas de átomos de semicondutores com \textit{gap} de banda maior para encapsular o núcleo das nanopartículas, formando então nanocristais núcleo-casca. Como consequência desta mudança, a eficiência luminescente foi melhorada e estudada por Hines e Guyot-Sionnest em 1996 na caracterização da alta luminescência de nanocristais de CdSe encapsulados em ZnS.

\par Ao longo dos anos, foram realizadas melhorias como encapsulação dupla do núcleo, introdução de ácidos mercapto-carboxílicos que possibilitaram a solubilidade e eficiência fotoluminescente em solventes orgânicos e água, além de proporcionar uma gama de aplicações dos pontos quânticos.

\par Para entender melhor como os pontos quânticos podem ser utilizados nestas diversas aplicaçõoes, é necessário o entendimento de conceitos relacionados a mecânica quântica e física do estado sólido, que serão apresentados nas seções a seguir.